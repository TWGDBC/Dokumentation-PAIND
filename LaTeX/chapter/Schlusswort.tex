\chapter{Reflektion}
\label{chap:Reflektion}

Die Aufgabenstellung konnte während der zur Verfügung stehenden Zeit nicht erfüllt werden. In den nachfolgenden Unterkapiteln wird Stellung genommen, aus welchen Gründen die definierten Ziele nicht erreicht wurden. Im Fazit werden Ergebnisse und Ziele gegenübergestellt und offene Punkte erläutert. Danach werden zum Projektmanagment einzelne Punkte aufgelistet mti Bezug zur detaillierten Projektplanung in Anhang \ref{Projektmanagment}. Der Ausblick bietet Aufschluss, welche weiteren Tätigkeiten zur Erfüllung der Aufgabenstellung geführt hätten. Zuletzt wird im Schlusswort eine persönliches Resümee gemacht.
\section{Fazit}
\label{sec:}

\section{Erläuterungen Projektmanagment}
\label{sec: pm}

Die Projektplanung wurde in der KW  aufgegliedert. Dabei gab es einige bedeutende Arbeitspakete, welche das Projektmanagment stark beinflusst haben. Die Einschätzung des zeitlichen Aufwands für die Erstellung der Software wurde falsch bewertet. Es musste bedeutend mehr Zeit für die Softwareentwicklung aufgewendet werden. Dies hat einerseits mit den ungenügenden Kenntnissen über die Programmiersprache C++ und dem Framework ROS zu tun. Anderseits konnten  Mehrere ROS-spezifische Fehlermeldungen, welche nur durch die Unterstürzung von Herr Jensen behoben werden konnten, nahmen viel Zeit in Anspruch.


\section{Ausblick}
\label{sec: Ausblick}

Für diese Aufgabenstellung 

\section{Schlusswort}
Aus persönlicher Sicht bin ich mit der geleisteten Arbeit zufrieden. Ich kann aus dieser Industriearbeit sehr viele positive Erkenntnisse herausziehen und werde diese in Hinsicht auf die nächste Arbeit einfliessen lassen. Das zeitliche Fortschreiten in der Realisierungsphase und einige Fehlüberlegungen führten zu einem nicht vollständigen Produkt.  

\section{Danksagung}
An dieser Stelle möchte ich mir herzlich bedanken, die mich bei der Anfertigung dieser Arbeit unterstützt haben.

Zuallererst gebührt der Dank an Dr. Björn Jensen, der mich bei dieser Industriearbeit tatkräftige unterstützt hat, sowie mit wertvollen Hinweisen und schnellen Rückmeldungen zur Seite gestanden ist.
Mein Dank geht auch an Jonas Räber, der mir eine grosse Hilfe für die Einarbeitung mit \ac{ROS} war. Ebenfalls bedanken möchte ich mich bei den zwei Gegenleser Andreas Zimmermann und Angela Burch für die textuelle und inhaltliche Analyse der Dokumentation.

Zuletzt noch besten Dank and D 

