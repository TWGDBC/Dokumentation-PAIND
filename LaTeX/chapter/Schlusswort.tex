\chapter{Reflektion}
\label{chap:Reflektion}

Die Aufgabenstellung konnte während der zur Verfügung stehenden Zeit nicht erfüllt werden. In den nachfolgenden Unterkapiteln wird Stellung genommen, aus welchen Gründen die definierten Ziele nicht erreicht wurden. Im Fazit werden Ergebnisse und Ziele gegenübergestellt und offene Punkte erläutert. Der Ausblick bietet Aufschluss, welche weiteren Tätigkeiten zur Erfüllung der Aufgabenstellung geführt hätten. Danach werden zum Projektmanagement einzelne Punkte aufgelistet mit Bezug zur detaillierten Projektplanung in Anhang \ref{Anhang}.  Zuletzt wird im Schlusswort ein persönliches Resümee gemacht.

\section{Fazit}
\label{sec:Ref_Fazit}
Die Aufgabenstellung bietet eine grosse Fülle an Disziplinen, welche in diesem Projekt abgedeckt werden. Dazu zählen: Recherche, Komponentenevaluation, Hardwarerealisierung, Softwareimplementation und Testversuche. Daher konnten nur begrenzt viel Zeit für die voneinander abhängenden Abschnitte aufgewendet werden. Zu Beginn der Arbeit waren keine Vorkenntnisse über Ubuntu, ROS und Koordinatentransformationen vorhanden. Diese drei Punkte erwiesen sich bei dieser Arbeit als die grössten Stolpersteine. Es musste viel Zeit aufgewendet werden, um  das Verständnis dafür zu erhalten. 
Durch falsche Komponentenwahl resultierten Fehlverhalten, welche zu Verzögerungen führten. Der Defekt des Gleichstrommotors konnte schnell behoben werden, hatte jedoch zur Folge, das mehrere Teilsysteme nicht mehr  korrespondierten. Die Schrittmotor bietet als Alternative guten Ersatz, da mit diesem schnell und unprolematisch Messungen gemacht und für die Aufgabenstellung Resultate ermöglicht wurden.

\section{Ausblick}
\label{sec: Ausblick}
Die Aufgabenstellung konnte nicht voll umfänglich gelöst werden. Es bietet sich an eine Verbesserung des bestehenden Prototyp in Ausblick zu stellen. Einerseits wurden mit der Analyse der state-of-the-art Projekte Grundlagen geschaffen, um ein Konkurrenz fähiges Produkt zu erstellen und anderseits konnte mit der mechanischen Konstruktion eine solche realisiert werden. Um das Lasermodul auch bei mobilen Anwendungen nutzbar zu machen, muss die aktuelle Konstruktion mit einer eigenen IMU erweitert werden. Da das Raspberry Pi bei den Tests bereits an die Grenzen der Leistungsfähigkeit gelangt, ist es ratsam ein leistungsfähigeren Einplatinencomputer, wie das Up Board Squared zu verwenden. Mit einem solchen Einplatinencomputer können in verschiedenen Bereichen Verbesserungen erreicht werden. Die Software bietet ein gutes Grundgerüst für die Erstellung von Punktwolken. Durch die mangelnde sensorischen Fähigkeiten kann diese jedoch nicht richtig eingesetzt werden. Eine Überarbeitung der aktuellen Sensorik für die endlos drehende Konstruktion ist naheliegend. 

\section{Erläuterungen Projektmanagment}
\label{sec: pm}

Die Projektplanung wurde in der Kalenderwoche 39 ausgearbeitet. Der zeitlichen Umfang einer Phase und der entsprechende Zeitaufwand wurde im Pflichtenheft abgeschätzt. Die abgeschätzten Phasen Initialisierung und Konzeptionsphase wurden gut abgeschätzt und die erwarteten Ergebnisse entsprechen den Erwartungen. Es musste mehr Zeit für die Informationsbeschaffung aufgewendet werden. Die Einarbeitung mit ROS benötigte bedeutende mehr Zeitaufwand. Dazu wurde auch zu viel Zeit aufgewendet, um für den Prototypen einen Einplatinencomputer auszuwählen. 

Die Realisierungsphase war zeitlich vehement umfangreicher als angenommen. Dabei gab es einige bedeutende Arbeitspakete, welche das Projektmanagement stark beeinflusst haben. Die Einschätzung des zeitlichen Aufwands für die Erstellung der Software wurde falsch bewertet. Es musste erheblich mehr Zeit für die Softwareentwicklung aufgewendet werden als eingeplant. Dies hat mit den ungenügenden Kenntnissen über die Programmiersprache C++/Python und dem Framework ROS zu tun. Die Projektphase wurde dadurch um eineinhalb Wochen erweitert. 

Dafür musste der Zeitaufwand für die Testphase gekürzt werden.
Die Dokumentation des gesamten Projekts war aufwändiger. Da während des gesamten Projekts die Dokumentation au-jour gehalten wurde, mussten Inhalte ständig auf neue Erkenntnisse und Veränderungen angepasst werden. 

Aus der anfänglichen Kalkulation geht eine Arbeitsaufwand von 218 Stunden hervor. Effektiv wurden durch die erläuterten zusätzlichen Aufwändigen 267.5 Stunden aufgewendet. Daraus resultiert ein Arbeitsüberschuss von 49.5 Stunden Mehraufwand.

\section{Schlusswort}
Aus persönlicher Sicht konnte bei dieser Arbeit nicht das volle Potenzial ausgeschöpft werden. Es wurde bei der Aufgabenstellung zu viele Wert auf ein funktionierendes Gesamtsystem gelegt, dass während der Konzeption zu wenig in der Funktionalität abgewogen wurde. Der Umfang der Arbeit war mit den anfänglich zur Verfügung stehenden Kompetenzen und Hilfestellungen sehr zeitraubend. die Konzipierung eines Gesamtsystems besitzt viele Finessen, welche ein breites Know-How benötigen. Es können aus dieser Industriearbeit sehr viele positive Erkenntnisse herausgezogen werden, um diese in Hinsicht auf die nächste Arbeit einfliessen zu lassen. Das zeitliche Fortschreiten in der Realisierungsphase und einige Fehlüberlegungen führten zu einem nicht vollständigen Prototyp. 

\section{Danksagung}
An dieser Stelle möchte ich mich bei allen bedanken, die mich bei der Ausführung dieser Arbeit unterstützt haben. Zuallererst gebührt der Dank an Dr. Björn Jensen, der mich bei dieser Industriearbeit tatkräftige unterstützt hat, mit wertvollen Hinweisen und schnellen Rückmeldungen zur Seite gestanden ist.
Mein Dank geht auch an den Assistenten Jonas Räber, der mir eine grosse Hilfe für die Einarbeitung mit \ac{ROS} war. Ebenfalls bedanken ich mich bei den Gegenlesern Andreas Zimmermann, Marie-Theres Zimmermann und Angela Burch für die syntaktische und inhaltliche Korrektur der wissenschaftlichen Dokumentation.

