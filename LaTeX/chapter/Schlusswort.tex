\chapter{Reflektion}
\label{chap:Reflektion}

Die Aufgabenstellung konnte während der zur Verfügung stehenden Zeit nicht erfüllt werden. In den nachfolgenden Unterkapiteln wird Stellung genommen, aus welchen Gründen die definierten Ziele nicht erreicht wurden. Im Fazit werden Ergebnisse und Ziele gegenübergestellt und offene Punkte erläutert. Danach werden zum Projektmanagment einzelne Punkte aufgelistet mit Bezug zur detaillierten Projektplanung in Anhang \ref{Projektmanagment}. Der Ausblick bietet Aufschluss, welche weiteren Tätigkeiten zur Erfüllung der Aufgabenstellung geführt hätten. Zuletzt wird im Schlusswort eine persönliches Resümee gemacht.

\section{Fazit}
\label{sec:Ref_Fazit}
Die Aufgabenstellung bietet eine grosse Fülle an Disziplinen, welche in diesem Projekt abgedeckt werden. Dazu zählen: Recherche, Komponentenorientiert, Hardwarerealisierung, Softwareimplementation und Testversuche. 
Zu Beginn der Arbeit waren keine Vorkenntnisse über ROS, Koordinatentransformation und C++ Softwareprojektierung vorhanden. Diese 3 Punkte erwiesen sich bei dieser Arbeit als die grössten Stolpersteine. Es musste viel Zeit aufgewendet werden, um ein grobes Verständnis  
Die Aufgabenstellung zielte auf eine Entwicklung eines Prototypen hin. Daher wurden die Zeiträume für die verschieden Projektphasen erstellt. 

\section{Erläuterungen Projektmanagment}
\label{sec: pm}

Die Projektplanung wurde in der KW  aufgegliedert. Dabei gab es einige bedeutende Arbeitspakete, welche das Projektmanagment stark beinflusst haben. Die Einschätzung des zeitlichen Aufwands für die Erstellung der Software wurde falsch bewertet. Es musste bedeutend mehr Zeit für die Softwareentwicklung aufgewendet werden als eingeplant. Dies hat einerseits mit den ungenügenden Kenntnissen über die Programmiersprache C++/Python und dem Framework ROS zu tun. Anderseits nahmen  mehrere ROS-spezifische Fehlermeldungen, welche nur durch die Unterstürzung von Herr Jensen behoben werden konnten, nahmen viel Zeit in Anspruch.


\section{Ausblick}
\label{sec: Ausblick}
Die Aufgabenstellung konnte nicht voll umfänglich gelöst werden. Es bietet sich an eine Verbesserung des bestehenden Prototyp in Ausblick zu stellen. Einerseits wurden mit der Analyse der State-of-the-art Projekte Grundlagen geschaffen, um ein Konkurrenz fähiges Produkt zu erstellen und anderseits konnte mit der mechanischen Konstruktion eine solche realisiert werden. Um das Lasermodul auch bei mobilen Anwendungen nutzbar zu machen, muss die aktuelle Konstruktion mit einer eigenen IMU erweitert werden. Da das Raspberry Pi bei den Tests bereits an die Grenzen der Leistungsfähigkeit gelangt, ist es ratsam ein leistungsfähigeren Einplatinencomputer, wie das Up Board Squred zu verwenden. Mit einem solchen Einplatinencomputer können in verschiedenen Bereichen Verbesserungen erreicht werden. Die Software bietet ein gutes Grundgerüst für die Möglichkeit von der Erstellung von PointClouds. Durch die mangelnde sensorischen Fähigkeiten müssen kann diese jedoch nicht richtig eingesetzt werden. Eine Überarbeitung der aktuellen Sensorik für die endlos drehende Konstruktion ist naheliegend.

\section{Schlusswort}
Aus persönlicher Sicht bin ich mit der geleisteten Arbeit zufrieden. Ich kann aus dieser Industriearbeit sehr viele positive Erkenntnisse herausziehen und werde diese in Hinsicht auf die nächste Arbeit einfliessen lassen. Das zeitliche Fortschreiten in der Realisierungsphase und einige Fehlüberlegungen führten zu einem nicht vollständigen Prototyp. Es konnte ein drehende Konstruktion realisiert werden, mit welcher eine Punktwolke erstellt werden kann.  

\section{Danksagung}
An dieser Stelle möchte ich mir herzlich bedanken, die mich bei der Anfertigung dieser Arbeit unterstützt haben.

Zuallererst gebührt der Dank an Dr. Björn Jensen, der mich bei dieser Industriearbeit tatkräftige unterstützt hat, sowie mit wertvollen Hinweisen und schnellen Rückmeldungen zur Seite gestanden ist.
Mein Dank geht auch an Jonas Räber, der mir eine grosse Hilfe für die Einarbeitung mit \ac{ROS} war. Ebenfalls bedanken möchte ich mich bei den zwei Gegenleser Andreas Zimmermann und Angela Burch für die textuelle und inhaltliche Analyse der Dokumentation.

Zuletzt noch besten Dank and die Gegenleser; Andreas Zimmermann, Marie-Theres Zimmermann und Angela Burch für die inhaltlichen und syntaktischen Korerktur dieser wissenschaftlichen Dokumentation.

