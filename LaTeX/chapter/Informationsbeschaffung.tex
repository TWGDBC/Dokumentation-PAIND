\chapter{Informationsbeschaffung}

Im Rahmen der Projektplanung wurde in einer ersten Phase ein Zeitraum zur Informationsbeschaffung festgelegt.

Nachfolgend werden die wichtigsten Erkenntnisse der Informationsbeschaffung erläutert, die maßgebend für die Konzeption in Kapitel \ref{Konzeption} und Realisierung in Kapitel \ref{Realisierung} sind. Dabei werden zu einzelnen Komponenten und Verfahren Stellung genommen und eruiert, ob diese sich für das Projekt eignen. Des weiteren sollen relevante Software erläutert werden, welche für die Realisierung nötig sind.


\section{bestehende Komponenten}
\label{bestehende_Sensoren}
In diesem Unterkapitel werden die bestehenden Komponenten Velodyne VLP-16 und der Hokuyo URG-LX01 detailliert betrachtet und die wichtigsten Spezifikation hervorgehoben.  

\subsection{Velodyne VLP-16 Puck}

Beim Velodyne VLP-16 Puck handelt es sich um einen Echtzeit 3D-Laser-Scanner, der auf dem LIDAR-Verfahren basiert. Dabei werden Der 3D Laserscanner 


\section{eruierte Komponenten}
\label{eruierte_Komponenten}


\section{ROS Robot Operating System}
Die gesamte Kommunikation mit Sensoren und Aktoren findet auf dem Packbot mit einem spezifisch implementierten Robot Operating System, kurz. ROS, statt. Daher ist es naheliegend, um die Integrität des zu erarbeitenden 3D-Laser-Moduls zu gewährleisten, dieses Software-Framework zu nutzen.  

\subscetion{ROS Kinetic Kame vs. Indigo} 
Grundsätzlich wird ROS auf einem Ubuntu Betriebssystem aufgesetzt und ist ein grösstenteils Kommando-basiertes Software-Framework. Diese in 2007 entwickelte Open Source Software erhielt in den letzten Jahren ständig neue und überarbeitete Versionen.  


\subsection 


\blinditemize

\section {weitere verwendete Software}


\subsection{Wireshark}

\label{Mind Map}




\section{Zwischenfazit}
\label{Zwischenfazit_Info}