\chapter{Einleitung}
\label{Einleitung}

\section{Blindtext} 
\blinditemize

\section {Projektauftrag}
\label{Projektauftrag}
%vIm Institut für INNOVATION UND TECHNOLOGIEMANAGEMENT IIT wird ständig mit dem mobilen Roboter "Packbot" an der Automatisierung und dessen Einsatzgebieten geforscht. Dabei werden immer wieder neue Technologien und Verfahren angewandt und damit neue Erkenntnisse zu erzielen. In diesem Zusammenhang steht nun der Projektauftrag ein 3D- Laserscanner Modul zu realisieren.   
.

\section{Aufgabenstellung}
\label{Aufgabenstellung}
Als Grundlage galt zum Zeitpunkt der Eingabe der Aufgabenstellung, dass ein 3D-Laser-Modul mit einem bestehenden 2D-Laser realisiert wird. Bei Projektbeginn im September 2017 wurde dies von Dr. Björn Jensen abgeändert, da nun ein 3D-Laserscanner zur Verfügung steht. Das zu e


Es soll ein 3D-Laser-Modul entwickelt werden, welches einen bestehenden 2D-Laser um eine Achse
rotiert und so die Vermessung der Umgebung in 3D erlaubt. Die gemessenen Distanzen sollen
von einem PC aufgenommen und dem mobilen Roboter einmal pro Umdrehung zur Verfügung gestellt
werden.
Üblicherweise bewegt sich der Roboter während diesen Messungen. Im Idealfall wird die Bewegung
des Roboters gemessen und die Messdaten entsprechend kompensiert.
Das entwickelte Laser-Modul soll im Rahmen der Arbeit auf dem Packbot-Roboter getestet werden.

d


\section{Ziele}
% ablgeich mit Projektziele Pflichtenheft
\label{Ziele}
Ziel des Projektes ist es die Realisierung eines 3D-Laser Moduls. In erster Priorität soll damit 3D Mapping in Echtzeit betrieben werden können. Das Modul wird mit dem bestehenden 3D Laserscanner von Velodyne des Typs VLP-16 realisiert. Dabei soll eine möglichst detaillierte Punktwolke erstellt werden, welche visualisiert werden kann. Zweite Priorität ist die Hinderniserkennung in Frontrichtung. Dazu muss in Frontrichtung eine detaillierte Punktwolke ermittelt werden können. Das Modul soll einerseits auf dem Packbot nutzbar, sowie auch eigenständig einsetzbar sein.