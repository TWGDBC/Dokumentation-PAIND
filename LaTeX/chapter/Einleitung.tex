
\chapter{Einleitung}

\section {Ausgangssituation}
\label{sec:Ausgangssituation}
Im Institut für Innovation und Technologiemanagment IIT an der Hochschule Luzern wird mit dem mobilen Roboter \grqq Packbot\grqq  im Bereich unbemannter Robotik stetig weitergeforscht. Dabei werden immer wieder neue Technologien und Verfahren angewandt und damit neue Erkenntnisse zu erzielen. Ein sehr aktuelles Thema ist dabei die digitale Kartographie. Mit präzisen Laserentfernungsmesser und entsprechender Software lassen sich heute detaillierte Karten von der Umgebung erstellen, sogenannt 3D-Mapping. Dabei werden die gemessen Distanzen zu einer Punktwolke (Pointcloud) zusammengeführt und visualisiert.
In diesem Zusammenhang steht nun die Aufgabenstellung des PAIND+E1.
.
\section{Aufgabenstellung}
\label{sec:Aufgabenstellung}
Nach Erhalt der Aufgabenstellung galt es anfänglich, ein 3D-Laser-Modul mit einem bestehenden 2D-Laser zu realisieren. Beim Projektbeginn im September 2017 wurde dies von Dr. Björn Jensen abgeändert, da nun ein 3D-Laserscanner für diese Aufgabe zur Verfügung steht. Beim zu erarbeitende Projekt handelt es sich um einen funktionsfähigen Prototypen. Der Prototyp soll sich um eine Achse drehen und die Daten dem mobilen Roboter einmal pro Umdrehung zur Verfügung stellen. Das entwickelte Laser-Modul soll im Rahmen der Arbeit auf dem Packbot-Roboter getestet werden.

\section{Ziele}
\todo{ablgeich mit Projektziele Pflichtenheft}
\label{sec:Ziele}
Ziel des Projektes ist die Realisierung eines 3D-Laser Moduls. Dabei wird die gesamte Hardware mit den gewählten Komponenten zusammengebaut. Die Software wird durch bestehende Codepakete und eigener Erweiterungen auf die Aufgabenstellung angepasst. In erster Priorität soll damit 3D Mapping in Echtzeit betrieben werden können. Das Modul wird mit dem bestehenden 3D-Laserscanner der Marke Velodyne des Typs VLP-16 realisiert. Dabei soll eine möglichst grosse räumliche Abdeckung der Umgebung erreicht werden. Diese wird in einer möglichst detaillierten Punktwolke visualisiert. Zweite Priorität ist die Hinderniserkennung in Frontrichtung. Dazu muss in Frontrichtung eine detaillierte Punktwolke ermittelt werden können. Das Modul soll einerseits auf dem Packbot nutzbar, sowie auch eigenständig einsetzbar sein.



\section{Methodik}
\label{sec:Methodik}
Für die Aufgabenstellung eignet sich eine strukturierter Projektphasenablauf. Dabei werden nacheinander die Phasen Initialisierung, Informationsbeschaffung, Konzeption, Realisierung und die Testphase durchlaufen. Das Pflichtenheft im Anhang A, grenzt die Aufgabenstellung und die definierten Phasen ab. Im Anhang B ist das dazugehörende Projektmanagement mit entsprechenden Erläuterungen angefügt. \todo{ref und evtl. Erweitern}
