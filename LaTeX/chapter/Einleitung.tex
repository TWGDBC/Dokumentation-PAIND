
\chapter{Einleitung}

Im Forschungszweig der Robotik entstehen aktuell ständig neue und revolutionäre Technologien. Dabei steht die Transformation des weitgehend selbstständigen Roboters im Vordergrund. Diverse Pilotprojekte beweisen bereits heute, dass durch eine komplexe Abstimmung hoch präziser Sensoren die kognitiven und sensorischen Fähigkeiten des Menschen nachgeahmt, wenn nicht sogar übertroffen werden können. Ein gutes Beispiel für diese Transformation sind mobile Roboter wie der iRobot Packpot. Durch entsprechende Logik und Sensorik können die geländegängigen Roboter dem Menschen einen enormen Dienst erweisen. Für Menschen unzugängliche oder nur unter hohem Gefahrenpotential begehbare Orte wie Kriegsgebiete, von Naturkatastrophen geschädigte oder radioaktiv verstrahlte Umgebungen, können sie Aufgaben bewältigen, welche dem Menschen alleine unmöglich erscheinen.

Durch die zunehmende Rechenleistung von Computern und den daraus resultierenden Datenmengen entsteht nun auch die Möglichkeit, mittels Robotern detaillierte Modelle der erwähnten Einsatzgebieten zu erstellen. An diesem Punkt setzt nun die Aufgabenstellung des PAIND+E1 an. Es soll ein low-cost Prototyp eines 3D-Laser-Modul entwickelt werden, mit dem eine dreidimensionales Modell der Umgebung möglichst detailliert visualisiert werden kann. Dabei soll einerseits die Frage geklärt werden, welche Konfiguration eine bestmögliche Modellierung der Umgebung bietet, und anderseits mit welchen Mitteln eine Realisierung möglich ist.

Nach Erhalt der Aufgabenstellung galt es anfänglich, ein 3D-Laser-Modul mit einem bestehenden 2D-Laser zu realisieren. Beim Projektbeginn im September 2017 wurde dies von Dr. Björn Jensen abgeändert, da nun ein 3D-Laserscanner für diese Aufgabe zur Verfügung stand. Der 3D-Laserscanner besitzt gegenüber dem 2D-Laser den Vorteil, dass bereits räumliche Messdaten zur Echtzeit übermittelt werden können. Zudem besitzt dieser auch einen bedeutend größeren Messbereich, welcher für die Erstellung von Umgebungsmodellen nötig ist. Beim zu erarbeitende Projekt handelt es sich um eine Realisierung eines funktionsfähigen Prototypen. Der Prototyp soll sich um eine Achse drehen und die Daten dem mobilen Roboter einmal pro Umdrehung zur Verfügung stellen. Das entwickelte Laser-Modul soll im Rahmen der Arbeit auf dem Packbot-Roboter getestet werden.

Ziel des Projektes ist die Realisierung eines 3D-Laser Moduls. Dabei wird die gesamte Hardware mit den gewählten Komponenten zusammengebaut. Die Software wird durch bestehende Codepakete und eigenen Erweiterungen auf die Aufgabenstellung angepasst. In erster Priorität soll damit 3D Mapping betrieben werden können. Das Modul wird mit dem bestehenden 3D-Laserscanner der Marke Velodyne des Typs VLP-16 realisiert. Dabei soll eine möglichst grosse räumliche Abdeckung der Umgebung erreicht werden. Diese wird in einer möglichst detaillierten Punktwolke modelliert. Zweite Priorität ist die Hinderniserkennung in Frontrichtung. Dazu muss in Frontrichtung eine detaillierte Punktwolke ermittelt werden können. Das Modul soll einerseits auf dem Packbot nutzbar, sowie auch eigenständig einsetzbar sein. Das Pflichtenheft im Anhang A, grenzt die Aufgabenstellung auf weitere Punkte ein. Alle Anhänge wurden in digitaler Form am Ende dieser Dokumentation auf einer CD hinterlegt. 

Für die Aufgabenstellung eignet sich ein strukturierter Projektphasenablauf. Dabei werden nacheinander die Phasen Initialisierung, Informationsbeschaffung, Konzeption, Realisierung und die Testphase durchlaufen. Im Anhang A Pflichtenheft sind anfänglich abgeschätzter Aufwand, Arbeitsmittel und die zu erwartende Ergebnisse beschrieben. Des Weiteren beinhaltet es die Vorgaben der Aufgabe mit entsprechenden Kriterien.
Der Inhalt der Dokumentation richtet sich nach den zu erarbeitenden Projektphasen. Da die Initialisierungsphase nur administrative Aufgaben beinhaltet wird diese Phase in dieser Dokumentation nicht näher erläutert. Die weiteren Phasen sind chronologisch mit entsprechenden Unterkapiteln im Inhaltsverzeichnis einsehbar. 

Im Anhang B ist ein detaillierter Projektplan angefügt, welcher die einzelne Arbeitspakete und das Zeitmanagement aufzeigt. Im Kapitel \ref{chap:Reflektion} werden dazu noch Erläuterungen zu Abweichungen, Problemstellungen und Zeitplanänderungen getätigt. Zusätzlich werden persönliche Reflektionen über das gesamte Projekt getätigt. 







