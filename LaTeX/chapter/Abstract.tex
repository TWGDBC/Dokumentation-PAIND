\chapter*{Abstract}
\label{chap:Abstract}
This documentation is the result of the project Modul PAIND+E1 at the Lucerne School of Engineering and Architecture for the industry partner RUAG AG. 

The following chapters contain the experiences and results during the project from September to December 2017. It describes the realisation of a 3D laser module prototype, which can map the environment. The main part is subdivided in different phases and reflects the timeline of the Project. 

The first part is a summary of the results during the information research phase. It contains the knowledge about the available 3D sensors, the potential hardware components and the necessary software to implement the solution after the functional specifications. 

A total of three concepts were developed, which have different approaches. The first concept called "platform", which can turn the 3D sensor in a wide range of angle, while using servo motors. This concept was no longer pursue, since the two other concepts were more suitable.

The two other concepts are based on a turning endlessly "tower". The idea to them are state-of-the-art projects. The difference between the two concepts is the position of the signal processing unit. In the unrotated version, the unit is below in a static case. Only the 3D-sensor is rotating for mapping. In the other version, the unit in the case is also rotating, only the interface is static. 

The main content is about the realised concept, which is similar to the unrotated version before. The realisation phase describes the process, how the case and the electronic parts are realised and mounted. In a separate topic, it describes, how the Software is implemented and how it works together with the hardware. 

After the realisation the prototype is tested. Because of a few faults and problems, the realised prototype can't satisfy the full requirements. 

In the end a short reflection summarised the largest challenges during the project and how to solve them. It also reflects the project management and give a little outlook.


\chapter*{Abstract}
Diese Dokumentation ist das Ergebnis des Projektmoduls PAIND+E1 an der Hochschule Luzern Technik und Architektur für den Industriepartner RUAG AG.

Die nachfolgenden Kapitel beinhalten Erfahrungen und Resultate des Projekts, die zwischen September bis Dezember 2017 erarbeitet wurden. Die Dokumentation beschreibt die Realisierung eines Prototyps, mit welchem die Umgebung vermessen und modelliert werden kann. Der Hauptteil wurde in verschiedene Projektphasen unterteilt und gibt den Verlauf des Projekts wieder.

Der erste Teil ist eine Zusammenfassung der Ergebnisse aus der Informationsbeschaffung. Es beinhaltet die wesentlichsten Spezifikationen des vorhandene 3D Sensor, die erforderlichen Hardwarekomponenten, sowie die nötige Software, um eine Lösung nach den Vorgaben zu entwickeln. 

Insgesamt wurden drei Konzepte erstellt. Das erste Konzept "Plattform" kann den 3D Sensor mittels Servomotoren in alle Richtungen neigen. Da sich nachfolgende Konzepte besser eignen, wurde dieses Konzept verworfen. Die zwei anderen Konzepte basieren auf einem endlos drehenden "Turm". Die Idee dazu liefern bestehende State-Of-The-Art Projekte. In der unrotierenden Version dreht sich der 3D Sensor mit dem Turm. Die Signalverarbeitung wird in einem statischen Gehäuse unterhalb des Sensors platziert. In der rotierenden Version dreht die gesamte Elektronik mit und nur die Schnittstellen werden unrotierend herausgeführt.

Der Hauptinhalt beschreibt das realisierte Konzept. Es handelt sich hierbei um eine angepasste unrotierende Version. Dabei wird der Bau des Gehäuses und der Einsatz der elektrischen Komponenten, sowie die Implemenation der Software erläutert.

Danch wurden die Funktionen des Prototyps getestet. Der erarbeite Prototyp konnte wegen einigen Fehler und Problemen die gewünschten Vorgaben nicht erfüllen. Zuletzt gibt eine Reflektion Auskunft über bedeutende Schwierigkeiten und Probleme, und wie sie gelöst wurden. Weiter wird zum Projektmanagement Stellung genommen und ein Ausblick für offene Punkte gegeben.





