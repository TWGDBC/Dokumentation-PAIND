\chapter*{Abstract}
\label{chap:Abstract}
This documentation is the result of the project Modul PAIND+E1 at the Lucerne School of Engineering and Architecture for the industry partner RUAG AG. The task was the realisation of a module, that maps the environment and creates a point cloud with the measured data. The 3D-sensor Velodyne VLP-16 is available for this purpose.

The following chapters contain the experiences and results during the project from September to December 2017. State-of-the-art projects have been investigated and compared. After that, components and software for implementation were analysed and evaluated.

A total of three concepts were elaborated, which have different approaches. The first concept turns the 3D sensor in a wide range of angle, while using servo motors. The two other concepts are based on a endlessly rotating "tower". The idea behind it, are the state-of-the-art projects. The difference between the two concepts is the position of the signal processing unit. In the unrotated version, the unit is below in a static case. Only the 3D-sensor is rotating for mapping. In the other version, the unit in the case is also rotating. Only the interface is static. 

The realised concept is similar to the unrotated version before. The realisation describes the process, how the case and the electronic parts are assembled. In a separate topic, it describes, how the Software with the Framework ROS is implemented and how it works together with the hardware. 

After the realisation the prototype was tested. Because of a few faults and problems, the realised prototype can't satisfy the full requirements.

Finally, this documentation offers approaches and considerations for creating a working 3D laser module.

\chapter*{Abstract}

Diese Dokumentation ist das Ergebnis des Projektmoduls PAIND+E1 an der Hochschule Luzern Technik und Architektur für den Industriepartner RUAG AG.

Die Aufgabe ist die Realisierung eines Moduls, mit dem die Umgebung vermessen und modelliert werden kann. Dazu steht der 3D-Sensor Velodyne VLP-16 zur Verfügung.

Die nachfolgenden Kapitel beinhalten Erfahrungen und Resultate des Projekts, die zwischen September bis Dezember 2017 erarbeitet wurden. Dafür wurden State-of-the-Art Projekte untersucht und verglichen. Danach wurden Komponenten und Software für die Realisierung gesucht und evaluiert.

Insgesamt wurden drei Konzepte erstellt. Das erste Konzept kann den 3D Sensor mittels Servomotoren in alle Richtungen neigen. Die zwei anderen Konzepte basieren auf einem endlos drehenden \glqq Turm\grqq. Die Idee dazu liefern die State-of-the-Art Projekte. In der unrotierenden Version dreht sich der 3D Sensor mit dem Turm. Die Signalverarbeitung wird in einem feststehenden Gehäuse unterhalb des Sensors platziert. In der rotierenden Version dreht die gesamte Elektronik mit und nur die Schnittstellen werden unrotierend herausgeführt.

Das realisierte Konzept ist eine angepasste unrotierende Version. Dabei wird der Bau des Gehäuses und der Einsatz der elektrischen Komponenten, sowie die Implemenation der Software mit dem ROS Framework erläutert.

Die Funktionen des Prototyps wurden ausgetestet. Der erarbeite Prototyp konnte wegen einigen Fehler und Problemen die gewünschten Vorgaben nicht erfüllen. 

Abschließend bietet diese Dokumentation Ansätze und Überlegungen, um einen funktionsfähigen 3D-Laser-Modul herzustellen.