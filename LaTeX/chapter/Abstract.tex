\chapter*{Abstract}
\label{chap:Abstract}
This Documentation is a result of the Project Modul PAIND+E1 at the Lucerne School of Engineering and Architecture for the industry partner RUAG AG written by Daniel Zimmermann. 

The following chapters contains the full experiences, results and descriptions during the project from September to Dezember 2017. The body of the documentation is subdivided in different phases and reflects the timeline of the Project. 

The first part is a summation of the results during the information research phase. It contains the knowledge about the available Sensors, the potential hardware components and the necessary Software to implement the solution after the functional specifications.  

There are three concepts created, which have different approaches. The first concept called "plattform" is based on a "regulation experiment", which can turn the plattform in a wide range of angle, while using servo motors. This concept was no longer pursue, because the two other concepts were more suitable.

The two other concepts are based on a turning endlessly "tower". The difference between the two concepts is the position of the signal processing unit. In the unrotaded version, the unit are below in a static case. Only the 3D-Sensor is rotating for mapping. In the rotated version, the signal processing unit in the case is also rotating, and only the Interface to the packpot is static. 

The main content is about the realisided concept, which is the last called concept before. The realisation phase is describes the process, how the case and the electornic parts are mounted. In a seperate topic, it describes, how the Software is implemented and how it works together with the Hardware. 

After the realisation, the modul is tested. There are a few Hard- and Software test protocols, which gives a feedback of the functionality and the outstanding problems.

In the end a short reflection summarised the largest challenges during the project and how to solve them. It also reflects the Project management and give a little outlook.



