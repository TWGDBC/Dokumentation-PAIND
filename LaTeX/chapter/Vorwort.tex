\chapter*{Vorwort}

In Zuge der industriellen Revolution 4.0, sog. digitale Revolution, entstehen gerade in der Robotik und Automation ständig neue und innovative Technologien. Dabei steht mehr und mehr die Transformation des weitgehend autonomen Roboter im Vordergrund. Dadurch dass heutzutage eine breite Palette an spezifischen Sensoren zur Verfügung stehen, kann die Virtualisierung und   

In der digitalen Vernetzung entlang der Wertschöpfungsketten und des gesamten Lebenszyklus von Marktleistungen liegt für die Schweizer Maschinen-, Elektro- und Metall-Industrie ein grosses wirtschaftliches Potenzial. Dieses reicht von Produktivitätssteigerungen über Innovationen bei Produkten und Dienstleistungen bis hin zu neuen Geschäftsmodellen. Die Schweizer Industrie verfügt über sehr gute Voraussetzungen für eine erfolgreiche Implementierung von Industrie 4.0. Swissmem engagiert sich zusammen mit drei weiteren Branchenverbänden in der Initiative «Industrie 2025», um das Thema in der Schweiz nachhaltig voranzutreiben und die digitale Transformation des Werkplatzes Schweiz zu fördern. 









Daniel Zimmermann, 22.10.2017