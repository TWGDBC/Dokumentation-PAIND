\chapter*{Vorwort}


In Zuge der industriellen Revolution 4.0, sog. digitale Revolution, entstehen gerade im Zweig der Robotik und der Automation ständig neue und bahnbrechende Technologien. Dabei steht die Transformation des weitgehend automatisierten Roboter im Vordergrund. Diverse Vorzeigeprojekte beweisen bereits heute, dass durch eine komplexe Abstimmung hoch präziser Sensoren die sensitiven und kognitiven Fähigkeiten des Menschen nachgeahmt, wenn nicht sogar übertroffen werden können.

Gerade für mobile Roboter wie der iRobot Packpot einen bedeutenden Beitrag leisten. Für Menschen unzugängliche oder nur unter hohem Gefahrenpotential zugängliche Orte wie Kriegsgebieten, von Naturkatastrophen geschädigten oder radioaktiv verstrahlten Umgebungen können sie Aufgaben bewältigen, welche dem Menschen alleine unmöglich erscheinen.

Diese Errungenschaften bieten in vielen Bereichen Einsatzmöglichkeiten, beispielsweise in der Industrie, der Medizin oder dem Transportwesen. Die Unterstützung eines automatisierten Systems kennt kaum Grenzen. 

Der Mensch kann mehr und mehr auf die Hilfe von mobilen Robotern zählen.


Daher ist es naheliegend, dass gerade bei der Kartographie eine dreidimensionale Visualisierung ein sehr interessantes und  Arbeitsgebiet bietet. 

Durch die zunehmende Rechenleistung von Computern und den daraus resultierenden Datenmengen enstehen Möglichkeiten







 







Daniel Zimmermann, 22.12.2017