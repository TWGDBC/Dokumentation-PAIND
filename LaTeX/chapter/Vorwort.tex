\chapter*{Vorwort}
\label{chap:Vorwort}

In Zuge der industriellen Revolution 4.0, sogenannte digitalen Revolution, entstehen gerade im Zweig der Robotik und der Automation ständig neue und revolutionäre Technologien. Dabei steht die Transformation des weitgehend automatisierten Roboter im Vordergrund. Diverse Vorzeigeprojekte beweisen bereits heute, dass durch eine komplexe Abstimmung hoch präziser Sensoren die kognitiven und sensorischen Fähigkeiten des Menschen nachgeahmt, wenn nicht sogar übertroffen werden können.

Ein gutes Beispiel für diese Transformation sind mobile Roboter wie der iRobot Packpot. Durch entsprechende Logik und Sensorik können die geländegängigen Roboter dem Menschen einen enormen Dienst erweisen. In für Menschen unzugängliche oder nur unter hohem Gefahrenpotential begehbare Orte wie Kriegsgebieten, von Naturkatastrophen geschädigten oder radioaktiv verstrahlten Umgebungen können sie Aufgaben bewältigen, welche dem Menschen alleine unmöglich erscheinen.

Durch die zunehmende Rechenleistung von Computern und den daraus resultierenden Datenmengen einsteht nun auch die Möglichkeiten mittels diesen unbemannten Robotern detailliert Visualisierungen in den erwähnten Einsatzgebieten zu erstellen. An diesem Punkt setzt nun die Aufgabenstellung des PAIND+E1 an. Es soll ein Prototyp eines 3D Laser Modul entwickelt werden, mit welchen eine 3D Karte der Umgebung möglichst detailliert visualisiert werden kann.

Daniel Zimmermann, 22.12.2017









 






