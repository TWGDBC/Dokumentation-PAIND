\begin{acronym}[LAMS aaaaaaaaa]
	  \acro{ARM}{Acorn Risc Machine}						\\	\acroextra{verbreitetes Mikroprozessor-Design, das grösstenteils auf 32Bit-Prozessoren beruht(ARMv7)}
	  \acro{AMD64}{X64}										\\	\acroextra{Prozessor-Architektur der handelsüblichen 64-Bit-PCs und Server}
 	  \acro{BSD}{Berkley Software Distribution}   			\\ \acroextra{frei verwendbare Lizenz, auch als Vorlage für kommerzielle Produkte}
 	  \acro{CPR}{Counts Per Revolution}						\\ \acroextra{}
 	  \acro{eMMC}{embedded Multimedia Card}   	 			\\ \acroextra{Digitales Speichermedium, basiert auf dem Prinzip der Flash Speicherung arbeitet}
  	  \acro{EnRicH}{European Robotic Hackathon}   	 		\\ \acroextra{weltweit erste und einzige Robotik Wettbewerb mit realen Szenarios}
 	  \acro{GPS}{Global Positioning System}      			\\ \acroextra{globales Navigationssatellitensystem zur Positionsbestimmung}
 	  \acro{GPU}{Grafikprozessor}				 			\\ \acroextra{spezialisierter Prozessor für Grafikanwendungen}
 	  \acro{IMU}{Inertiale Messeinheit}      				\\ \acroextra{Kombination mehrerer Trägheitssensoren}
      \acro{LIDAR}{Light Detection And Ranging}			 	\\ \acroextra{Verfahren zur optischen Distanzmessung}
      \acro{Mikrostepping}{Schrittteilung}					\\ \acroextra{Unteteilung der zu tätigenden Schritten bie Schrittmotoren um feinere Abstufungen zu erhalten}
      \acro{PCL}{Point Cloud Library}						\\ \acroextra{umfangreiche Softwarebibliothek für 3D Visualisierungen}
      \acro{KO}{Kathodenoszilloskop}		 				\\	\acroextra{Messgerät zur Analyse von Signalen}
      \acro{PWM}{Pulsweitenmodulation}						 \\ \acroextra{Modulationsverfahren mit varierenden Rechteckimpulsen}
      \acro{Quaternionen}{Erweiterung der komplexen Zahlen}								\\ \acroextra{Beschreibung von Drehbewegungen, bietet den Vorteil von eindeutigen 3D Positionen}
      \acro{RAM}{Random Acceess Memory}						\\ \acroextra{direkt ansprechbare Speicherbausteine}
      \acro{Refactoring}{ Strukturverbesserung}				\\ strukturelle Anpassung, verbessert Lesbarkeit, Wartbarkeit und Erweiterbarkeit
      \acro{ROS}{Robot Operating System}					\\ \acroextra{Software Framework für Robotikanwendungen}
      \acro{Scope}{Sichbarkeitsbereich}						\\ \acroextra{Sichtbarkeitsbereich einer Variable in einem Softwareprojekt}
      \acro{SD}{Secure Disk Memory Card}					\\ \acroextra{digitales Speichermedium, das nach dem Prinzip der Flash-Speicherung arbeitet}
      \acro{SOC}{Sysem-on-a-Chip}						 	\\ \acroextra{Integration der Funktionen eines programmierbaren elektronischen Systems auf einem Chip}
      \acro{State-Machine}{sequenzielle Ablauffolge}		\\ \acroextra{Verhaltensmodell einer Software, dass Aktionen, Zuständen und Zusandsänderungen beschreibt}
       \acro{UDP}{User Data Protocol}						\\ \acroextra{verbindungsloses Netzwerkprotokoll zur Versendung von Datagrammen von IP-basierten Rechennetzen} 
        \acro{XPC}{Barebone PC}								\\ \acroextra{kompakte unvollständige Computerdesign, meist mit vielen Schnittstellen und guter Erweiterbarkeit}
      
\end{acronym} 
