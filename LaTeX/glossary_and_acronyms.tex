\begin{acronym}[LAMS aaaaaaaaa]
	  \acro{ARM}{Acorn Risc Machine}					\\	 \acroextra{verbreitetes Mikroprozessor-Design, das grösstenteils32Bit-Prozessoren beruht ()ARMv7)}
	  \acro{AMD64}{X64}					\acroextra{Prozessor-Architektur der handelsüblichen 64-Bit-PC und Server}
 	  \acro{BLAM}{Berkeley Localization And Mapping} \\ \acroextra{ist ein Open-Source Softwarepaket für LiDAR-basierte Echtzeit 3D}
 	  \acro{BSD}{Berkley Software Distribution}   	 \\ \acroextra{frei verwendbare Lizenz, auch als Vorlage für kommerzielle Produkte}
 	  \acro{CPR}{Counts Per Revolution}
 	  \acro{eMMC}{embedded Multimedia Card}   	 \\ \acroextra{Digitales Speichermedium, das nach dem Prinzip der Flash Speicherung arbeitet}
  	  \acro{EnRicH}{European Robotic Hackathon}   	 \\ \acroextra{weltweit erste und einzige Robotik Wettbewerb mit realen Szenarios}
 	  \acro{GPS}{Global Positioning System}      	\\ \acroextra{globales Navigationssatellitensystem zur Positionsbestimmung}
 	  \acro{GPU}{Grafikprozessor}				 \\ \acroextra{spezialisierter Prozessor für Graphikanwendungen}
 	  \acro{IMU}{Inertiale Messeinheit}      	\\ \acroextra{ ist eine Kombination mehrerer Trägheitssensoren}
      \acro{LIDAR}{Light Detection And Ranging}			 \\ \acroextra{Verfahren zur optischen Distanzmessung}
      \acro{KO}{Kathodenoszilloskop}		 	\\	\acroextra{Messgerät zur Analyse von Signalen}
      \acro{PWM}{Pulsweitenmodulation}					 \\ \acroextra{Verfahren}
      \acro{RAM}{Random Acceess Memory}					 \\ \acroextra{Verfahren}
      \acro{ROS}{Robot Operating System}				 \\ \acroextra{Verfahren}
      \acro{SD}{Secure Disk Memory Card}					 \\ \acroextra{digitales Speichermedium, das nach dem Prinzip der Flash-Speicherung arbeitet}
      \acro{SLAM}{Simultaneous Localisation and Mapping}		 \\ \acroextra{Verfahren}
      \acro{SOC}{Sysem-on-a-Chip}						 \\ \acroextra{Integration der Funktionen eines programmierbaren elektronischen Systems auf einem Chip}
       \acro{UDP}{User Data Protocol}				 \\ \acroextra{verbindungsloses Netzwerkprotokoll zur Versendung von Datagrammen von IP-basierten Rechennetzen} 
        \acro{XPC}{Barebone PC}				 \\ \acroextra{kompakte unvollständige Computerdesign, meist mit vielen Schnittstellen und guter Erweiterbarkeit}
      
\end{acronym} 
